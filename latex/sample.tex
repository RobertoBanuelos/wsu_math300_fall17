\documentclass{article}

\usepackage{fullpage}

%%
%% common packages for math from the AMS
%%
\usepackage{amssymb}
\usepackage{amsmath}
\usepackage{amsthm}

%%
%% defining environments for some common math parts
%%
\newtheorem{definition}{Definition}
\newtheorem{theorem}{Theorem}
\newtheorem{corollary}{Corollary}

\begin{document}

\begin{theorem}[First derivative test for a local extremum]
Let $p$ be a critical point of $f$, and suppose $f'$ changes sign in going through $p$.   Then $f$ has a strict local extremum at $p$.  The extremum is a maximum if $f'$ changes sign from plus to minus, and a minimum if $f'$ changes sign from minus to plus.
\end{theorem}

\begin{proof}

Suppose $f'$ changes sign from plus to minus in going through 
$p$, in a deleted $\delta$-neighborhood of $p$, that is, in the 
union of intervals $(p-\delta, p) \cup (p, p+\delta)$.
By the mean value theorem in increment form,

\[ f(p + \Delta x) - f(p) = f'(p + \alpha \Delta x) \Delta x \quad (0 < \alpha < 1). \]

\noindent Therefore

\begin{equation}
\label{eq:1}
f(p + \Delta x) - f(p) < 0
\end{equation}

\noindent if $-\delta < \Delta x < 0$, since then $f'(p + \alpha \Delta x) > 0$, $\Delta x < 0$, and similarly

\begin{equation}
\label{eq:2}
f(p + \Delta x) - f(p) < 0
\end{equation}

\noindent if $0 < \Delta x < \delta$, since then $f'(p + \alpha \Delta x) < 0$, $\Delta x > 0$.  In other words,

\begin{equation}
\label{eq:3}
f(p + \Delta x) < f(p)
\end{equation}

\noindent if $0 < |\Delta x| < \delta$, so that $f$ has a strict local maximum
at $p$.  On the other hnad, if $f'$ changes sign from minus to plus in
going through $p$, then we get $>$ instead of $<$ in (\ref{eq:1}),
(\ref{eq:2}), and (\ref{eq:3}), so that $f$ has a strict local minimum
at $p$ (check the details).

\end{proof}

\newpage

\noindent \textbf{Euler's formula:}

\[ e^{iz} = \cos z + i \sin z \quad \mbox{ and } \quad e^{-iz} = \cos z - i \sin z \]

\newpage

\noindent Complex $n^{th}$ root of unity:

\[ \omega \in \mathbb{C} \mbox{ such that } \omega^n = 1 \]

\noindent There are exactly $n$ complex $n^{th}$ roots of unity:

\[ e^{2 \pi i \frac{k}{n}} \quad k=0,1, \cdots, n-1 \] 

\noindent Recall:

\[ e^{i u} = \cos(u) + i \sin(u) \]

\newpage

\noindent Integrals:

\begin{equation}
\int_1^{Z} \frac{1}{z} \, dz = \log Z
\end{equation}

\noindent Limits.  Let $y = f(x)$.  Since $f$ is differentiable
at $x$ and $g$ is differentable at $y = f(x)$, both limits:

\begin{eqnarray*}
f'(x) & = & \lim_{\Delta x \to 0} \frac{f(x + \Delta x) - f(x)}{\Delta x}, \\
g'(x) & = & \lim_{\Delta y \to 0} \frac{g(y + \Delta y) - g(y)}{\Delta y}
\end{eqnarray*}

\noindent exist, but then

\begin{eqnarray*}
\lim_{\Delta x \to 0} \left[ \frac{f(x + \Delta x) - f(x)}{\Delta x} - f'(x) \right] & = & 0 \\
\lim_{\Delta y \to 0} \left[ \frac{g(y + \Delta y) - g(y)}{\Delta y} - g'(y) \right] & = & 0 \\
\end{eqnarray*}

\newpage

\noindent Sets:

\[ \mathcal{X} = \bigcap_{s_i \in \mathcal{S}} s_i \]

\noindent Sums:

\[ \sum_{i=1}^{N} i^2 \]

\noindent The sum $\sum_{i=1}^{N} i^2$ is big.

\end{document}
