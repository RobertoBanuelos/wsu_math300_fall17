\documentclass[12pt]{article}

\usepackage{fullpage}
\usepackage{url}
\usepackage{nopageno}

\title{MATH 300: Mathematical Computing (3 cr.)}
\author{Matthew Sottile (\texttt{matthew.sottile@wsu.edu})}
\date{Fall 2017}

\begin{document}

\maketitle

\section*{Logistics}

\noindent \textbf{Course prerequisite:} MATH 220 or MATH 230

\noindent \textbf{Meeting schedule:} 4:15-5:30 Tues/Thurs

\noindent \textbf{Office:} TBD

\noindent \textbf{Office Hours:} By request via Google hangouts


\noindent \textbf{Required text materials:}
None

\section*{Student learning outcomes and assessment}

\begin{itemize}

%%
%%

\item \textbf{Outcome:} communicate mathematics \emph{effectively} using mathematical
typesetting languages

\textbf{Topics to address outcome:} \LaTeX

\textbf{Assessment:} Homework assignments, tests.  Evaluation will focus on
mathematical components of \LaTeX.

%%
%%

\item \textbf{Outcome:} communicate mathematics effectively using the world wide web

\textbf{Topics to address outcome:} Markdown, basic HTML, github pages

\textbf{Assessment:} homework assignments, questions on quizzes/tests.

%%
%%

\item \textbf{Outcome:} basic understanding of modern networked computing concepts.

\textbf{Topics to address outcome:} discussion of DNS, TCP/IP, computer arithmetic and number systems

\textbf{Assessment:} quiz and test questions.

%%
%%

\item \textbf{Outcome:} understand the rudiments of writing mathematical computer programs.

\textbf{Topics to address outcome:} Python, as well as at least one other mathematical programming language (Julia, Octave).

\textbf{Assessment:} Homework assignments, questions on quizes and tests.

%%
%%

\item \textbf{Outcome:} write mathematics well

\textbf{Topics to address outcome:} discussion and demonstration of mathematical writing styles and expectations.

\textbf{Assessment:} Several assignments include significant writing components.

\end{itemize}

\section*{Expectations for student effort}

You should expect to spend 2-5 hours per week on homework and test
preparation.

\section*{Week to week course outline}

Table~\ref{fig:schedule} is a guideline only. We will spend more or less time on
topics as required by student needs, and quizzes and tests will be
scheduled in consultation with students.

\begin{figure}
\centering
\begin{tabular}{|r||c|c|} \hline
Week & Tuesday & Thursday \\ \hline
1 & --- & INTRO \\
2 & \LaTeX (syllabus) & \LaTeX (tables, layout) \\
3 & \LaTeX (equations) & \LaTeX (figures, code) \\
4 & QUIZ, \LaTeX (bibtex) & \LaTeX (beamer) \\
5 & MARKDOWN, GH Pages & Git \\
6 & Networking concepts & Computer arithmetic \\
7 & Language for math overview & Python (basics) \\
8 & TEST & Python (types; lists; objects) \\
9 & Python (OO) & Python (Functional) \\
10 & Python (Numpy) & Python (Sympy) \\
11 & Python (Jupyter) & Python (Plotting) \\
12 & Quiz, OCTAVE & OCTAVE \\
13 & JULIA & JULIA \\
14 & MAXIMA & Theorem provers \\
15 & MATHEMATICA & MATHEMATICA \\
16 & TEST & --- \\ \hline
\end{tabular}
\caption{Approximate schedule.}
\label{fig:schedule}
\end{figure}

\section*{Description of required assignments}

There will be two tests worth a total of 200 points. In addition there
will be two quizzes worth 50 points each. There will be several
assignments worth 400-460 points.  These will typically fall into one of a few distinct categories.

\begin{itemize}
\item Typeset some document for which we provide an image as a template.

\item Write some functions according to strict specifications.

\item Solve some open-ended mathematical problem, and write about the
solution and how you found it.  These assignments are much more
extensive than the other classes of problems, and involve a
substantial writing component.
\end{itemize}

\section*{Grading policy:}

\begin{figure}[h!]
\centering
\begin{tabular}{|c|c|} \hline
Percentage & Guaranteed Grade \\ \hline
93 &	A \\
90 &	A- \\
87 &	B+ \\
83 &	B \\
80 &	B- \\
77 &	C+ \\ 
73 &	C \\
70 &	C- \\
60 &	D \\
0 &	F \\ \hline
\end{tabular}
\caption{Grading policy.}
\label{fig:grading}
\end{figure}


The policy (shown in Figure~\ref{fig:grading}) regarding grades is to line percentages up in numerical
order, and draw lines between grades in the gaps. This means that if
your friend has a 93.01\% she will get an A, but in that situation, if
you have 92.97\%, you will probably also get an A. To summarize, scores
that are very close will receive the same grade. I also reserve the
right to be merciful: there have been semesters when scores as low as
91\% still received an A because I felt that a test contained a
question that was misunderstood, or more difficult than I had
anticipated.

\section*{Late assignments}

Assignments are turned in electronically. There will always be a
deadline for an assignment, but an assignment coming in after the
deadline receives no penalty until after the first batch of
assignments is graded. Any assignments received after the initial
bunch is graded, but before they have been returned, incurs a 10%
penalty. Once graded assignments are returned, those who missed the
deadline can still submit the assignment, but their work receives a
20\% penalty. After solutions to an assignment are posted, then no
further submissions are accepted.

\section*{Attendance policy}

None. We are adults now. Class time is useful for a variety of
reasons, but if you disagree then you are free to use your time as you
find most valuable.

\section*{WSU reasonable accomodations statement}

``Students with Disabilities: Reasonable accommodations are available
for students with a documented disability. If you have a disability
and need accommodations to fully participate in this class, please
either visit or call the Access Center at 509-335-3417, Washington
Building 217; http://accesscenter.wsu.edu, Access.Center@wsu.edu to
schedule an appointment with an Access Advisor. All accommodations
MUST be approved through the Access Center.''

\section*{WSU academic integrity statement}

``Academic integrity is the cornerstone of higher education. As such,
all members of the university community share responsibility for
maintaining and promoting the principles of integrity in all
activities, including academic integrity and honest
scholarship. Academic integrity will be strongly enforced in this
course. Students who violate WSU's Academic Integrity Policy
(identified in Washington Administrative Code (WAC) 504-26-010(3) and
-404) will receive scores of zero on on the assignment or test in
question. they will not have the option to withdraw from the course
pending an appeal, and will be reported to the Office of Student
Conduct. Cheating includes, but is not limited to, plagiarism and
unauthorized collaboration as defined in the Standards of Conduct for
Students, WAC 504-26-010(3). You need to read and understand all of
the definitions of cheating. If you have any questions about what is
and is not allowed in this course, you should ask course instructors
before proceeding. If you wish to appeal a faculty member's decision
relating to academic integrity, please use the form available at
conduct.wsu.edu.''

\section*{Course statement on collaboration}

Collaboration is one of the best ways to learn, and I encourage it on
assignments. I can tell the difference between collaboration and
plagiarism when your work and that of your collaborator are
substantially different; when I witness you and your collaborator
working together; and perhaps most importantly, when you identify your
collaborator in your work. Put comments in your code; put
acknowledgements in your papers. No collaboration is permitted on
tests or quizzes.

\section*{Safety and emergency notification}

Classroom and campus safety are of paramount importance at Washington
State University, and are the shared responsibility of the entire
campus population. WSU urges students to follow the "Alert, Assess,
Act," protocol for all types of emergencies and the "Run, Hide, Fight"
response for an active shooter incident. Remain ALERT (through direct
observation or emergency notification), ASSESS your specific
situation, and ACT in the most appropriate way to assure your own
safety (and the safety of others if you are able).

Please sign up for emergency alerts on your account at MyWSU. For more
information on this subject, campus safety, and related topics, please
view the FBI's Run, Hide, Fight video and visit the WSU safety portal.


\end{document}
